% Jacob Neumann

% DOCUMENT CLASS AND PACKAGE USE
\documentclass[12pt]{article}
    \usepackage{spec}
    \definecolor{auxColor}{HTML}{804020}
    \definecolor{mainColor}{HTML}{e48020}
    \definecolor{brightColor}{HTML}{ffefe0}

\title{Regexp Module}
\author{Jacob Neumann}
\date{July 2021}

\graphicspath{{../img/}}

\newcommand{\Vals}{\textsf{Values}}
\begin{document}

\maketitle

\tableofcontents

\clearpage

\begin{note}
    Throughout, \code{Sigma} is assumed to be some equality type.
\end{note}
\begin{note}
    This document assumes the definitions and declarations of \url{https://github.com/smlhelp/aux-library/blob/main/documentation/language.pdf}, and assumes that the \code{Language} structure has been opened.
\end{note}
In particular, we have
\begin{definition}
    \auxLib{Language.sml}{31}{31}

    A total function \code{D : Sigma list -> bool} is said to \keyword{compute} a set $L$ of values of type \code{Sigma list} if 
        \[ \code{D cs}\stepsTo\code{true} \qtq{iff} \code{cs}\in L \]
    i.e. we can run \code{D} on any value \code{cs : Sigma list} and get back either \code{true} or \code{false}, which correctly indicates that either $\code{cs}\in L$ or $\code{cs}\not\in L$, respectively. 
\end{definition}


\clearpage
\section{Preliminary Declarations \& Main Definitions}
\begin{declaration}
    \auxLib{Regexp.sml}{29}{35}
\end{declaration}

\begin{definition}[Language]
    Given a value \code{R : Sigma regexp}, define the \keyword{language of \code R} to be the set of values $\mc L(\code R)\sub \Vals(\code{Sigma list})$ given recursively by:
\begin{alignat*}{2}
    &\mc L(\code{Zero}) \quad &&= \quad \emptyset\\ 
    &\mc L(\code{One}) \quad &&= \quad \set{\code{[]}}\\ 
    &\mc L(\code{Const(c)}) \quad &&= \quad \set{\code{[c]}}\\ 
    &\mc L(\code{Plus(r1,r2)}) \quad &&= \quad \mc L(\code{r1})\cup \mc L(\code{r2})\\ 
    &\mc L(\code{Times(r1,r2)}) \quad &&= \quad \compSet{\code v_1\code{@v}_2}{\code v_1\in\mc L(\code{r1})\text{ and }\code v_2\in\mc L(\code{r2})}\\ 
    &\mc L(\code{Star(r)}) \quad &&= \quad \compSet{\code v_1\code{@v}_2\code{@ ... @v}_n}{n\in\N,\ \code v_1,\code v_2,\ldots,\code v_n\in\mc L(\code r)}
\end{alignat*}
\end{definition}

\begin{definition}
    A total function \code{D : Sigma regexp -> Sigma language} is said to \keyword{compute the regular expression decision problem} (over the alphabet \code{Sigma}) if
        \[ \code{D R cs}\stepsTo\code{true} \qtq{if and only if} \code{cs}\in\mc L(\code R) \]
    for all \code{R : Sigma regexp} and all \code{cs : Sigma list}.
\end{definition}

\begin{definition}
    A pair \code{(p,s) : Sigma list * Sigma list} is said to be a \keyword{splitting} of \code{cs : Sigma list} if
    \[ \code{cs} \eeq \code{p @ s}. \]
    In such a splitting, \code{p} is called the \keyword{prefix} and \code{s} the \keyword{suffix}.
\end{definition}

\begin{declaration}
    \auxLib{Regexp.sml}{47}{47}
\end{declaration}

\begin{definition}\label{defn:almost-total}
    For any type \code{t}, we'll say that a function 
        \[ \code{k : Sigma list * Sigma list -> t} \]
    is \keyword{almost total} if, for all \code{(p,s) : Sigma list * Sigma list}, either
        \[ \code{k(p,s)}\evalsTo\code{v}\text{ for some value \code{v} \quad\quad or \quad\quad} \code{k(p,s)}\text{ raises }\code{NoMatch}. \]

    If $\code{k(p,s)}\evalsTo\code{v}$, then we say that  \keyword{\code{k} accepts \code{(p,s)} with result \code{v}}.
\end{definition}
\clearpage
\section{Matching Specification}

\begin{declaration}[Depth]
    Define the \keyword{depth} of \code{R : Sigma regexp} recursively by
    \auxLibIndent{Regexp.sml}{37}{45}{-20pt}
\end{declaration}

\Spec{match}{%
    \code{''a regexp}\\
    &\quad\quad\code{-> ''a list}\\
    &\quad\quad\code{ -> (''a list * ''a list -> 'b)}\\
    &\quad\quad\code{-> 'b}}
{\code{k} is almost total (\autoref{defn:almost-total})}
{    \[ \code{match r cs k} \eeq \begin{cases}
    \code{v} & \parbox[t]{.43\textwidth}{where \code{(p,s)} is a splitting of \code{cs} such that $\code p\in\mc L(\code r)$ and \code{k} accepts \code{(p,s)} with result \code v.}\\
                                    \code{raise NoMatch} & \text{if there is no such \code{(p,s)}}
                                \end{cases}
    \]
}

\spec{LL}{''S regexp -> ''S language}{}{\code{LL} computes the regular expression decision problem: for all \code{R : Sigma regexp}, \code{(LL R) : Sigma list -> bool} is a total function such that
    \[ \code{LL R cs}\stepsTo\code{true} \qtq{iff} \code{cs}\in\mc L(\code R) \]} 

\clearpage
\section{Matching Implementation}
\auxLib{Regexp.sml}{29}{47}
\clearpage
\auxLib{Regexp.sml}{49}{75}
\clearpage


\section{\texttt{Zero}, \texttt{One}, \texttt{Const} Correctness}

\begin{lemma}\label{lemma:zero-correct}
    For any almost total \code{k} and any \code{cs : Sigma list},
\[ \code{match Zero cs k} \eeq \begin{cases}
    \code{v} & \parbox[t]{.43\textwidth}{where \code{(p,s)} is a splitting of \code{cs} such that $\code p\in\mc L(\code{Zero})$ and \code{k} accepts \code{(p,s)} with result \code v.}\\
                                    \code{raise NoMatch} & \text{if there is no such \code{(p,s)}}
                                \end{cases}
    \]
\end{lemma}
\begin{proof}
    Since $\mc L(\code{Zero})=\emptyset$, there cannot be any splitting of any \code{cs} whose prefix is in $\mc L(\code{Zero})$. Therefore, this spec says that \code{match Zero cs k} always ought to raise \code{NoMatch}, which it does. \specSat
\end{proof}
\begin{corollary}\label{cor:zero-almost-total}
    For any almost total \code{k}, \code{(fn cs => match Zero cs k)} is almost total.
\end{corollary}

\begin{lemma}\label{lemma:one-correct}
    For any almost total \code{k} and any \code{cs : Sigma list},
\[ \code{match One cs k} \eeq \begin{cases}
    \code{v} & \parbox[t]{.43\textwidth}{where \code{(p,s)} is a splitting of \code{cs} such that $\code p\in\mc L(\code{One})$ and \code{k} accepts \code{(p,s)} with result \code v.}\\
                                    \code{raise NoMatch} & \text{if there is no such \code{(p,s)}}
                                \end{cases}
    \]
\end{lemma}
\begin{proof}
    Since $\mc L(\code{One})=\set{\code{[]}}$, the only splitting of \code{cs} which we need to consider is \code{([],cs)}. But notice that
        \[ \code{k([],cs)} \qeq \begin{cases}
                                    \code{v} &\text{if \code{k} accepts \code{([],cs)} with result \code{v}}\\
                                    \code{raise NoMatch} &\text{if \code{k} rejects \code{([],cs)}}
                                \end{cases}
        \]
    so we have \specSat by putting \code{match One cs k = k([],cs)}.
\end{proof}
\begin{corollary}\label{cor:one-almost-total}
    For any almost total \code{k}, \code{(fn cs => match One cs k)} is almost total.
\end{corollary}


\begin{lemma}\label{lemma:const-correct}
    For any almost total \code{k}, any \code{cs : Sigma list}, and any value \code{c : Sigma}
    \[ \code{match (Const c) cs k} \eeq \begin{cases}
    \code{v} & \parbox[t]{.35\textwidth}{where \code{(p,s)} is a splitting of \code{cs} such that $\code p\in\mc L(\code{Const c})$ and \code{k} accepts \code{(p,s)} with result \code v.}\\
                                    \code{raise NoMatch} & \text{if there is no such \code{(p,s)}}
                                \end{cases}
    \]
\end{lemma}
\begin{proof}
    Begin by recalling that $\mc L(\code{Const c}) = \set{\code{[c]}}$. 
    Break into two cases: \code{cs=[]} and \code{cs=c'::cs'}.

    If \code{cs=[]}, then there is no prefix of \code{cs} in $\mc L(\code{Const c})$, so \code{match (Const c) [] k} should raise \code{NoMatch}. \specSat

    If \code{cs=c'::cs'}, then break into two further cases: either \code{c=c'} or \code{c<>c'}. In the latter case, there is no prefix of \code{cs} in $\mc L(\code{Const c})$, so again \code{NoMatch} should be raised. \specSat However, if \code{c=c'}, then $\code{[c']}\in \mc L(\code{Const c})$, so whatever behavior \code{k([c'],cs')} has is what behavior \code{match (Const c) cs k} should have. \specSat
\end{proof}
\begin{corollary}\label{cor:const-almost-total}
    For any almost total \code{k} and any value \code{c:Sigma}, \code{(fn cs => match (Const c) cs k)} is almost total.
\end{corollary}
%\[ \code{match r cs k} \eeq \begin{cases}
%    \code{v} & \parbox[t]{.43\textwidth}{where \code{(p,s)} is a splitting of \code{cs} such that $\code p\in\mc L(\code r)$ and \code{k} accepts \code{(p,s)} with result \code v.}\\
%                                    \code{raise NoMatch} & \text{if there is no such \code{(p,s)}}
%                                \end{cases}
%    \]


\section{\texttt{Plus} Correctness}
\inductive{For some \code{R1,R2 : Sigma regexp},
      \[ \code{match R1 cs k} \eeq \begin{cases}
          \code{v} & \parbox[t]{.43\textwidth}{where \code{(p,s)} is a splitting of \code{cs} such that $\code p\in\mc L(\code{R1})$ and \code{k} accepts \code{(p,s)} with result \code v.}\\
                                    \code{raise NoMatch} & \text{if there is no such \code{(p,s)}}
                                \end{cases}
    \]     
      \[ \code{match R2 cs k} \eeq \begin{cases}
          \code{v} & \parbox[t]{.43\textwidth}{where \code{(p,s)} is a splitting of \code{cs} such that $\code p\in\mc L(\code{R2})$ and \code{k} accepts \code{(p,s)} with result \code v.}\\
                                    \code{raise NoMatch} & \text{if there is no such \code{(p,s)}}
                                \end{cases}
    \]
    for all values \code{cs : Sigma list} and all almost total \code{k}.

    In particular: if \code{k} is almost total, then so too are \code{(fn cs => match R1 cs k)} and \code{(fn cs => match R2 cs k)}.
    }

\begin{lemma}\label{lemma:plus-correct}
    For any almost total \code{k}, any \code{cs : Sigma list},
    \[ \code{match (Plus(R1,R2)) cs k} \eeq \begin{cases}
        \code{v} & \parbox[t]{.3\textwidth}{where \code{(p,s)} is a splitting of \code{cs} such that $\code p\in\mc L(\code{Plus(R1,R2)})$ and \code{k} accepts \code{(p,s)} with result \code v.}\\
                                    \code{raise NoMatch} & \text{if there is no such \code{(p,s)}}
                                \end{cases}
    \]
\end{lemma}
%\begin{proof}

%\end{proof}
\begin{corollary}\label{cor:plus-almost-total}
    For any almost total \code{k}, \code{(fn cs => match (Plus(R1,R2)) cs k)} is almost total.
\end{corollary}

\section{\texttt{Times} Correctness}
\inductive{For some \code{R1,R2 : Sigma regexp},
      \[ \code{match R1 cs k} \eeq \begin{cases}
          \code{v} & \parbox[t]{.43\textwidth}{where \code{(p,s)} is a splitting of \code{cs} such that $\code p\in\mc L(\code{R1})$ and \code{k} accepts \code{(p,s)} with result \code v.}\\
                                    \code{raise NoMatch} & \text{if there is no such \code{(p,s)}}
                                \end{cases}
    \]     
      \[ \code{match R2 cs k} \eeq \begin{cases}
          \code{v} & \parbox[t]{.43\textwidth}{where \code{(p,s)} is a splitting of \code{cs} such that $\code p\in\mc L(\code{R2})$ and \code{k} accepts \code{(p,s)} with result \code v.}\\
                                    \code{raise NoMatch} & \text{if there is no such \code{(p,s)}}
                                \end{cases}
    \]
    for all values \code{cs : Sigma list} and all almost total \code{k}.

    In particular: if \code{k} is almost total, then so too are \code{(fn cs => match R1 cs k)} and \code{(fn cs => match R2 cs k)}.
    }

\begin{lemma}\label{lemma:times-correct}
    For any almost total \code{k}, any \code{cs : Sigma list},
    \[ \code{match (Times(R1,R2)) cs k} \eeq \begin{cases}
        \code{v} & \parbox[t]{.3\textwidth}{where \code{(p,s)} is a splitting of \code{cs} such that $\code p\in\mc L(\code{Times(R1,R2)})$ and \code{k} accepts \code{(p,s)} with result \code v.}\\
                                    \code{raise NoMatch} & \text{if there is no such \code{(p,s)}}
                                \end{cases}
    \]
\end{lemma}
%\begin{proof}

%\end{proof}
\begin{corollary}\label{cor:times-almost-total}
    For any almost total \code{k}, \code{(fn cs => match (Times(R1,R2)) cs k)} is almost total.
\end{corollary}

\section{\texttt{Star} Correctness}
\inductive{For some \code{R : Sigma regexp} and some \code{cs : Sigma list}
      \[ \code{match R cs g} \eeq \begin{cases}
          \code{v} & \parbox[t]{.43\textwidth}{where \code{(p,s)} is a splitting of \code{cs} such that $\code p\in\mc L(\code{R})$ and \code{g} accepts \code{(p,s)} with result \code v.}\\
                                    \code{raise NoMatch} & \text{if there is no such \code{(p,s)}}
                                \end{cases}
    \]     
      \[ \code{match (Star(R)) cs' g} \eeq \begin{cases}
          \code{v} & \parbox[t]{.43\textwidth}{where \code{(p,s)} is a splitting of \code{cs'} such that $\code p\in\mc L(\code{Star(R)})$ and \code{g} accepts \code{(p,s)} with result \code v.}\\
                                    \code{raise NoMatch} & \text{if there is no such \code{(p,s)}}
                                \end{cases}
    \]
    for all values \code{cs' : Sigma list} \textit{of length less than \code{cs}}, and all almost total \code{g}.

    In particular: if \code{g} is almost total, then so too is \code{(fn cs => match R cs g)}. And if \code{g} is almost total, \code{match (Star(R)) cs' g} for \code{cs'} of length less than \code{cs}.
    }

\begin{lemma}\label{lemma:star-correct}
    For any almost total \code{k},
    \[ \code{match (Star(R)) cs k} \eeq \begin{cases}
        \code{v} & \parbox[t]{.3\textwidth}{where \code{(p,s)} is a splitting of \code{cs} such that $\code p\in\mc L(\code{Star(R)})$ and \code{k} accepts \code{(p,s)} with result \code v.}\\
                                    \code{raise NoMatch} & \text{if there is no such \code{(p,s)}}
                                \end{cases}
    \]
\end{lemma}
%\begin{proof}

%\end{proof}
\begin{corollary}\label{cor:star-almost-total}
    For any almost total \code{k}, \code{(fn cs => match (Star(R)) cs k)} is almost total.
\end{corollary}


\section{\texttt{match} and \texttt{LL} Correctness}

\section{Reduction, Representation, and Printing Specification}
\spec{represent}{(''S -> string) -> ''S regexp -> string}{\code{toStr} is total, \code{R} is non-\code{Zero}}{\code{represent toStr R} converts the regular expression \code{R} into POSIX-like regular expression syntax, using \code{toStr} to convert \code{Sigma} values into strings. This uses the convention of representing \code{One} as \texttt{(.\{0,0\})}}

\begin{definition}
    A value \code{R : Sigma regexp} is said to be \keyword{\code{Zero}-reduced} if either:
    \begin{itemize}
        \item \code{R = Zero}
        \item \code{R} does not contain any \code{Zero}s
    \end{itemize}
\end{definition}

\spec{reduce}{''S regexp -> ''s regexp}{}{$\code{(reduce R)}\stepsTo\code{R'}$ such that $\mc L(\code R)=\mc L(\code{R'})$ and \code{R'} is \code{Zero}-reduced (and also \code{R'} may have some superfluous \code{One}s removed)}

\specEffect{printRep}{(''S -> string) -> ''S regexp -> unit}{\code{toStr} is total, $\code{(reduce R)}\not\stepsTo\code{Zero}$}{$\code{printRep toStr R}\stepsTo\code{()}$}{Prints the value of \code{(represent toStr (reduce R))} into the REPL.}


\end{document}
\section{(Almost) Totality}
\auxBox{Prop.}  For any equality type \code{T}, any type \code{t}, any value \code{r : T regexp}, any value \code{cs : T list}, and any value \code{k : T list * T list -> t} satisfying the REQUIRES clause for \code{match}, either
    \[ \code{match r cs k}\evalsTo\code{v}\text{ for some }\code v \qtq{or} \code{match r cs k}\text{ raises }\code{NoMatch} \]
\begin{proof}
    Let \code{k} be an arbitrary value satisfying the REQUIRES. We proceed by structural induction on \code{r}.

    \noindent\auxBox{Base Case: \texttt{Const}} \code{r=Const(a)} for some \code{a:T}. If \code{cs=[]} or \code{cs=c::cs'} for some \code{c<>a}, then observe
        \[ \code{match r cs k}\qeq\code{raise NoMatch}. \]
        Otherwise, if $\code{cs}\eeq\code{a::cs'}$, then
        \[ \code{match r cs k}\quad\evalsTo\quad \code{k([a],cs')} \]
        and, by hypothesis, \code{k} either evaluates to a value or raises \code{NoMatch}.
   
    \noindent\auxBox{Base Case: \texttt{Zero}} \code{r=Zero}. Observe \code{match Zero cs k} always raises \code{NoMatch}, satisfying the claim.
   
    \noindent\auxBox{Base Case: \texttt{One}} \code{r=One}.
        \[ \code{match One cs k}\quad\evalsTo\quad\code{k([],cs)} \]
        and, by hypothesis, \code{k} either evaluates to a value or raises \code{NoMatch}.
       
    \noindent\auxBox{Inductive Hypothesis} For all \code{g} satisfying the REQUIRES, and all \code{cs},
    \begin{align*}
        \code{match r1 cs g}\evalsTo\code{v}\text{ for some }\code v \qtq{or} \code{match r1 cs g}\text{ raises }\code{NoMatch} \\ 
        \code{match r2 cs g}\evalsTo\code{v}\text{ for some }\code v \qtq{or} \code{match r2 cs g}\text{ raises }\code{NoMatch}
    \end{align*}
    \noindent\auxBox{Inductive Step: \code{Times}} \code{r=Times(r1,r2)}. 
     Notice that since \code{k} satisfies the REQUIRES, the function \code{k''} given by
        \[ \code{fn (res'',cs'') => k(res'@res'',cs'')} \]
    also satisfies the REQUIRES (assuming \code{res'} is some value). We can use this fact, plus our inductive hypothesis for \code{r2}, to get that the function \code{k'} given by
        \[ \code{fn (res',cs') => match r2 cs' k''} \]
    always either evaluates to a value or raises \code{NoMatch} when applied to some value \code{(res',cs')}. Thus \code{k'} satisfies the REQUIRES. So, by the inductive hypothesis for \code{r1},
        \[ \code{match r1 cs k'}\evalsTo\code{v}\text{ for some }\code v \qtq{or} \code{match r1 cs k'}\text{ raises }\code{NoMatch} \]
        and since $\code{match r cs k}\stepsTo\code{match r1 cs k'}$, we have the claim.

    \noindent\auxBox{Inductive Step: \code{Plus}} \code{r=Plus(r1,r2)}. By the inductive hypothesis for \code{r1}, we have that \code{match r1 cs k} either evaluates to a value or raises \code{NoMatch}. If it raises \code{NoMatch}, then
        \[ \code{match r cs k}\qstepq\code{match r2 cs k}\]
        and, by the inductive hypothesis for \code{r2}, this either evaluates to a value or raises \code{NoMatch}.
    
    \noindent\auxBox{Inductive Step: \code{Star}} \code{r=Star(r1)}. By the assumption that \code{k} satisfies the REQUIRES, the evaluation of \code{k([],cs)} either evaluates to a value or raises \code{NoMatch}. If it evaluates to a value \code{v}, then
        \[ \code{match (Star(r1)) cs k}\quad\evalsTo\quad\code{v} \]
        too. If it raises \code{NoMatch}, then
        \[ \code{match r cs k}\qstepq \parbox[t]{0.5\textwidth}{
                \code{match r cs (fn (res',cs') =>}\\ 
                \code{if (cs = cs')}\\
                \code{then raise NoMatch}\\
                \code{else}\\
                         \code{match (Star(r)) cs' (fn (res'',cs'') => 
                             k(res'@res'',cs'')))}}\]
        Now observe that the continuation
        \[ \parbox[t]{0.85\textwidth}{
            \code{(fn (res',cs') =>}\\ 
            \code{\ \ \ \ if (cs = cs')}\\
            \code{\ \ \ \ then raise NoMatch}\\
            \code{\ \ \ \ else match (Star(r)) cs' (fn (res'',cs'') =>}\\
            \code{\ \ \ \ \ \ \ \ k(res'@res'',cs'')))}} \]
        either evaluates to a value or raises \code{NoMatch} for any \code{(res',cs')}: if \code{cs=cs'}, then this raises \code{NoMatch}. Otherwise, it steps to
        \[ \code{match (Star(r)) cs' (fn (res'',cs'') => k(res'@res'',cs''))}. \]
        Since \code{k} evaluates to a value or raises \code{NoMatch}, so does \code{(fn (res'',cs'') => k(res'@res'',cs''))}, and so t
\end{proof}
 


\section{Correctness}

\auxBox{Defn.} Given \code{r : T regexp} and \code{k} satisfying the REQUIRES of \code{match}, a pair of \code{T list} values \code{(p,s)} is said to \keyword{satisfy} \code{r} and \code{k} if
    \[ \code{p}\in\mc L(\code{r}) \qtq{and} \code{k(p,s)}\evalsTo\code{v}\text{ for some value }\code{v} \]

\auxBox{Thm.} For any equality type \code{T}, any type \code{t}, any value \code{cs : T list}, and any value \code{k : T list * T list -> t} satisfying the REQUIRES clause for \code{match},
    \[  \code{match r cs k} 
        \eeq  \begin{cases}
            \code{k(p,s)} & \parbox[t]{.35\textwidth}{where \code{(p,s)} is a splitting of \code{cs} that satisfies \code{r} and \code{k}}\\
                \code{raise NoMatch} & \text{if there is no such \code{(p,s)}}
              \end{cases}
    \]
\begin{proof}
    Proceed by structural induction on \code{r}.

    \noindent\auxBox{Base Case: \texttt{Const}} \code{r=Const(a)} for some \code{a:T}. Pick arbitrary \code k satisfying the REQUIRES. We split into two cases, \code{cs=[]} and \code{cs=c::cs'}. If \code{cs=[]}, then the only possible prefix of \code{cs} is \code{[]}. However,
        \[ \code{[]} \not\in \mc L(\code{Const(a)}), \]
    so no splitting of \code{cs} can satisfy \code{r} and \code{k}. Accordingly,
        \begin{align*} 
            \code{match (Const(a)) [] k} \qeq \code{raise NoMatch}. \tag*{\specSat}
        \end{align*}

    If \code{cs=c::cs'}, then either \code{c=a} or \code{c<>a}. If \code{c=a}, then \code{[c]} is the only prefix of \code{cs} which is in $\mc L(\code{Const(a)})$. Since
        \begin{align*}
            \code{match (Const(a)) (c::cs') k}\qstepq\code{k([c],cs')} \tag{when \code{c=a}}
        \end{align*}
        we get that \code{match r cs k} will evaluate to whatever value \code{k(p,s)} evaluates to if \code{([c],cs')} satisfies \code{r} and \code{k}, or raise \code{NoMatch} if \code{k([c],cs')} does. \specSat If \code{c<>a}, then no prefix of \code{cs} is in $\mc L(\code r)$, so no splitting of \code{cs} satisfies \code{r} and \code{k}. So \code{match r cs k} raises \code{NoMatch}. \specSat
    
    \noindent\auxBox{Base Case: \texttt{Zero}} \code{r=Zero}. Pick arbitrary \code{cs} and arbitrary \code{k} satisfying the REQUIRES. There is no prefix of \code{cs} in $\mc L(\code{r})=\emptyset$, so no splitting of \code{cs} satisfies \code{r} and \code{k}. Therefore,
        \begin{align*}
            \code{match Zero cs k}\qeq \code{raise NoMatch}. \tag*{\specSat}
        \end{align*}
    
    \noindent\auxBox{Base Case: \texttt{One}} \code{r=One}. For any \code{cs}, the prefix \code{[]} is in $\mc L(\code{One})$, and this is the only prefix of \code{cs} in $\mc L(\code{One})$. Therefore, the only splitting of \code{cs} that could possibly satisfy \code{r} and \code{k} is \code{([],cs)}. So
        \[ \code{match One cs k}\qstepq\code{k([],cs)} \]
    so \code{match r cs k} evaluates to the same value as \code{k([],cs)} if \code{([],cs)} satisfies \code{r} and \code{k}, and raises \code{NoMatch} if not. \specSat

    \noindent\auxBox{Inductive Hypothesis} For all \code{k} satisfying the REQUIRES, and all \code{cs},
    \begin{align*}
        \code{match r1 cs k} 
        &\eeq  \begin{cases}
            \code{k(p,s)} & \parbox[t]{.35\textwidth}{where \code{(p,s)} is a splitting of \code{cs} that satisfies \code{r1} and \code{k}}\\
                \code{raise NoMatch} & \text{if there is no such \code{(p,s)}}
              \end{cases}\\
      \code{match r2 cs k} 
        &\eeq  \begin{cases}
            \code{k(p,s)} & \parbox[t]{.35\textwidth}{where \code{(p,s)} is a splitting of \code{cs} that satisfies \code{r2} and \code{k}}\\
                \code{raise NoMatch} & \text{if there is no such \code{(p,s)}}
              \end{cases}
    \end{align*}

    \noindent\auxBox{Inductive Step: \code{Times}} \code{r=Times(r1,r2)}. 
    
    Let \code{k} be any function satisfying the REQUIRES.
    %First, suppose there is some splitting \code{(p,s)} of \code{cs} that satisfies \code{r} and \code{k}. This means:
    %\begin{itemize}
    %    \item $\code{p}\in\mc L(\code{Times(r1,r2)})$, i.e. there exists $\code{p1}\in\mc L(\code{r1})$ and $\code{p2}\in\mc L(\code{r2})$ such that $\code{p}\cong\code{p1@p2}$
    %    \item \code{k(p,s)} reduces to a value
    %\end{itemize}


        Suppose there is a splitting \code{(p,s)} of \code{cs} satisfying \code{Times(r1,r2)} and \code{k}. Then, by definition, there must be $\code{p1}\in\mc L(\code{r1})$ and $\code{p2}\in\mc L(\code{r2})$ such that $\code{p}\eeq\code{p1@p2}$ and \code{k(p,s)} evaluates to some value. So then observe 
            \[ \code{(p1,p2@s)}\text{ satisfies \code{r1} and \code{k'}}. \]
        To see this, note that $\code{p1}\in\mc L(\code{r1})$ and that
        \begin{align*}
            &\code{k'(p1,p2@s)}\\
                &\stepsTo \code{match r2 cs' k''} \tag{$\code{p2@s}\evalsTo\code{cs'}$}\\
                &\eeq \begin{cases}
                \code{k''(p'',s'')} & \parbox[t]{.35\textwidth}{where \code{(p'',s'')} is a splitting of \code{cs'} that satisfies \code{r2} and \code{k''}}\\
                \code{raise NoMatch} & \text{if there is no such \code{(p'',s'')}}
              \end{cases}
        \end{align*}
        but there is such a \code{(p'',s'')}, namely \code{(p2,s)}. This is because we specified that $\code{p2@s}\eeq\code{cs'}$, we assumed that $\code{p2}\in\mc L(\code{r2})$ and we know \code{k''(p2,s)} evaluates to a value:
        \begin{align*}
            \code{k''(p2,s)} &\stepsTo \code{k(p1@p2,s)}\\
                             &\eeq \code{k(p,s)} \tag{$\code{p1@p2}\eeq\code{p}$, above}\\
                             &\evalsTo\text{ some value} \tag{assumed above}
        \end{align*}
        So, in this case, we get that
            \[ \code{match r cs k} \qeq \code{k(p,s)}. \]
        \specSat

        Now assume there is no such splitting of \code{cs} which satisfies \code{Times(r1,r2)} and \code{k}. This could be the case because there is no prefix of \code{cs} in $\mc L(\code{r1})$. In this case, there cannot be any \code{(p',s')} satisfying \code{r1} and \code{k}, so we'll get
            \begin{align*}
                \code{match r cs k}\qeq \code{raise NoMatch}. \tag*{\specSat}
            \end{align*}
        So suppose \code{p1} is some prefix of \code{cs} which is in $\mc L(\code{r1})$.

    \noindent\auxBox{Inductive Step: \code{Plus}}
    
    \noindent\auxBox{Inductive Step: \code{Star}}
\end{proof}

\end{document}
\begin{itemize}
    \item Does typechecking occur before, after, or at the same time as evaluation? \Blank{before}
    \item Can an ill-typed expression be evaluated? \Blank{No}
    \item How is the type of an expression determined?
        \stBlank\ifsol
            Recursively, based on the types of its subexpressions.
        \fi\enBlank
\end{itemize}

\section{Extensional Equivalence}

\auxBox{Defn.} Well-typed expressions \code{e} and \code{e'} are said to be \keyword{extensionally equivalent} (written $\code{e}\eeq\code{e'}$) if \Blank[6cm]{they have the same type} and either:
\begin{itemize}
    \item \Blank[10cm]{\code{e} and \code{e'} evaluate down to the same value \code{v}},
    \item \Blank[10cm]{the evaluation of \code{e} and \code{e'} both loop forever}, or
    \item \Blank[10cm]{the evaluation of \code{e} and \code{e'} both raise the same exception}.
\end{itemize}
If I have a chain of extensional equivalences of the form
    \begin{align*}
        \code{e}_{\code 1} 
            &\eeq \code{e}_{\code 2}\\
            &\eeq \code{e}_{\code 3}\\
            &\;\;\vdots\\
            &\eeq \code{e}_{\code n}
    \end{align*}
why am I allowed to conclude that $\code{e}_{\code n}\cong\code{e}_{\code 1}$?
\stBlank\ifsol
    Because $\cong$ is an equivalence relation, so is symmetric and transitive.
\fi\enBlank
\ifsol \else \clearpage \fi
True or false:
\begin{itemize}
    \item \Blank{false} if $\code{e}\cong\code{e'}$ then $\code{e}\stepsTo\code{e'}$.
    \item \Blank{true} if $\code{e}\stepsTo\code{e'}$ then $\code{e}\cong\code{e'}$.
    \item \Blank{true} if $\code{e}\cong\code{e'}$, then \code{e} is valuable if and only if \code{e'} is.
\end{itemize}

\section{Binding \& Scope}

Write the SML syntax for making a binding of the integer value 5 to the identifier \code{x} (don't forget type annotations!)
\stCodeBlank\ifsol
    \begin{codeblock}
        val x : int = 5
    \end{codeblock}
\fi\enBlank

Consider the following piece of SML code
\begin{codeblock}
    val a : string = "world"
    val b : string = "hello " ^ a
    val a : string = "there"
    val c : string = "(" ^ b ^ ")"
\end{codeblock}
\begin{itemize}
    \item Notice that we make a binding to \code{a} on both the first and third lines. What's it called when we do that? \Blank{shadowing}
    \item Should you ever do that in your own code? \Blank{No}
    \item What value gets bound to \code{c}? \Blank{\code{"(hello world)"}}
    \item Explain your answer to the previous blank
        \stBlank\ifsol
            When evaluating the expression \code{"hello " ^ a} on the second line, the value bound to \code{a} in the current environment is \code{"world"}, so \code{"hello " ^ a} will evaluate to \code{"hello world"}, and that value will get bound to \code{b}. When evaluating the expression on the last line, this is still the most recent binding to \code{b}, even though we've shadowed \code{a}. So \code{"hello world"} will be used for \code{b} in \code{"(" ^ b ^ ")"}, and thus \code{"(hello world)"} will be the value bound to \code{c}.
        \fi\enBlank
\end{itemize}

\clearpage

\section{Function Application}
What does it mean if a value \code{v} has type \code{T1 -> T2} for some types \code{T1}, \code{T2}?
\stBlank\ifsol
    It means that \code{v} is a function, which accepts arguments of type \code{T1} and outputs values of type \code{T2}
\fi\enBlank
When is an expression of the form
\begin{codeblock}
    e1 e2
\end{codeblock}
well-typed? What's its type?
\stBlank\ifsol
    This expression is well-typed if there are types \code{T} and \code{T'} such that \code{e1} is a well-typed expression of type \code{T -> T'} and \code{e2} is a well-typed expression of type \code{T}. If so, then the whole expression, \code{e1 e2}, will have type \code{T'}.
\fi\enBlank

\end{document}
