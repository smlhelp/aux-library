% Jacob Neumann

% DOCUMENT CLASS AND PACKAGE USE
\documentclass[12pt]{article}
    \usepackage{spec}

    \definecolor{mainColor}{HTML}{be3f81}
    \definecolor{auxColor}{HTML}{662245}
    \definecolor{brightColor}{HTML}{ffe0e0}

\title{Language Module}
\date{July 2021}

\author{Jacob Neumann}

\graphicspath{{../img/}}

\begin{document}

\maketitle

\tableofcontents

\clearpage

\newcommand{\Vals}{\textsf{Values}}

\section{Decision Problems}

\begin{definition}[Set of values]
    For any type \code{t}, write $\Vals(\code t)$ to denote the set of all values of type \code{t} (up to extensional equivalence).
\end{definition}
\begin{example}
    \begin{itemize}
        \item $\Vals(\code{bool}) = \set{\code{true}, \code{false}}$
        \item $\Vals(\code{bool -> bool}) = \set{\code{not}, \code{Fn.id}, \code{(fn _ => true)}, \code{(fn _ => false)}}$
    \end{itemize}
\end{example}
\begin{definition}
    A \keyword{decision problem} of type \code{t} is a subset 
        \[ L \sub \Vals(\code t) \]
\end{definition}
\begin{definition}\label{defn:computes}
    A decision problem $L\sub\Vals(\code t)$ is said to be \keyword{decidable} if there is a \textbf{total} value \code{D : t -> bool} such that 
                \[ \code{D(v)}\stepsTo\code{true} \qtq{iff} \code{v}\in L \]
            for all values \code{v:t}. In this case, we say that \code{D} \keyword{decides} (or \keyword{computes}) $L$
\end{definition}
\begin{example}
    The \textit{subset sum problem} is a family of decision problems of type \code{int list}, one for each value \code{n : int}
    \[ \textsf{SUBSETSUM}_{\code n} = \compSet{\code{l}\in\Vals(\code{int list})}{\code{foldr op+ 0 l}\eeq \code{n}} \]
    For each \code{n}, the problem $\textsf{SUBSETSUM}_{\code n}$ is decidable: we can define a total function
    \[ \code{subsetSum n : int list -> bool} \]
    such that $\code{subsetSum n l}\stepsTo\code{true}$ iff $\code{foldr op+ 0 l}\eeq\code{n}$.
\end{example}

\begin{definition}\label{defn:language}
    Given a total function \code{D : t -> bool}, define the \keyword{language} of \code{D} to be the set
    \[ \mc L(\code{D}) = \compSet{\code v\in\Vals(\code t)}{\code{D(v)}\stepsTo\code{true}}. \]
\end{definition}


\begin{definition}
    An \keyword{equality type} is any type \code{t} such that
    \[ \code{ (op =) : t * t -> bool } \]
    is well-typed, i.e. any type whose values we can compare with the \code{=} and \code{<>} operators.
\end{definition}

\begin{definition}
    Given an equality type \code{Sigma}, a \keyword{decision problem over the alphabet \code{Sigma}} is a subset
        \[ L \sub \Vals(\code{Sigma list}). \]
\end{definition}

\section{Specification}

\begin{note}
    Throughout, \code{Sigma} will denote some equality type, the type of our ``alphabet''. Though some of our results will hold when \code{Sigma} is allowed to be a general polymorphic type (e.g. \autoref{lemma:boolAlg}), we are mainly concerned with situations where \code{Sigma} is an equality type (and the values \code{singleton} and \code{just} demand that \code{Sigma} be an equality type).
    
    A paradigm example is to take
    \begin{codeblock}
    type Sigma = char
    \end{codeblock}

    
\end{note}

\begin{lemma}
    For any total function \code{D : Sigma list -> bool} and any subset $L\sub\Vals(\code{Sigma list})$, the following are equivalent:
    \begin{enumerate}
        \item[(1)] \code{D} computes $L$ (\autoref{defn:computes})
        \item[(2)] For \textit{all} values \code{v : Sigma list},
            \[ \code{D(v)} \quad\stepsTo\quad \begin{cases}
                                                    \code{true} &\text{ if }\code{v}\in L\\
                                                    \code{false} &\text{ if }\code v\not\in L
                                              \end{cases}
            \]
        \item $\mc L(\code D) = L$ (\autoref{defn:language})
    \end{enumerate}
\end{lemma}
%\begin{note}
%    A total function \code{f : t -> bool} decides $L$ iff $\mc L(\code f)=L$.
%\end{note}
\TypeSpec{'S language = 'S list -> bool}{}{}

\spec{everything}{'S language}{}{\code{everything : Sigma language} is a total function computing the decision problem $\Vals(\code{Sigma list})$ over the alphabet \code{Sigma}.}
\spec{nothing}{'S language}{}{\code{nothing : Sigma language} is a total function computing the decision problem $\emptyset$.}
\spec{singleton}{''S list -> ''S language}{}{\code{(singleton v): Sigma language} is a total function computing the decision problem $\set{\code v}$}
\spec{just}{''S list list -> ''S language}{}{$\code{(just [v}_1,\code{v}_2,\ldots,\code{v}_n\code{]): Sigma language}$ is a total function computing the decision problem $\set{\code v_1,\code v_2,\ldots,\code v_n}$}

\spec{Or}{'S language * 'S language -> 'S language}{\code{L1} and \code{L2} are total}{\code{Or(L1,L2)} is a total function such that
    \[ \mc L(\code{Or(L1,L2)}) = \mc L(\code{L1}) \cup \mc L(\code{L2}) \]
}
\spec{And}{'S language * 'S language -> 'S language}{\code{L1} and \code{L2} are total}{\code{And(L1,L2)} is a total function such that
    \[ \mc L(\code{And(L1,L2)}) = \mc L(\code{L1}) \cap \mc L(\code{L2}) \]
}
\spec{Not}{'S language -> 'S language}{\code{L} is total}{For any \code{L : Sigma language}, \code{Not(L)} is a total function such that
    \[ \mc L(\code{Not(L)}) = \Vals(\code{Sigma list}) \setminus \mc L(\code{L}) \]
}
\spec{Xor}{'S language * 'S language -> 'S language}{\code{L1} and \code{L2} are total}{\code{Xor(L1,L2)} is a total function such that
    \[ \mc L(\code{Xor(L1,L2)}) = \mc L(\code{Or(L1,L2)}) \setminus \mc L(\code{And(L1,L2)}) \]
}

\spec{lengthEqual}{int -> 'S language}{$\code{n}\geq\code0$}{\code{lengthEqual n} is a total function such that
    \begin{align*}
        & \mc L(\code{lengthEqual n}) \\
        & = \compSet{\code{s}\in\Vals(\code{Sigma list})}{\code{((List.length s)=n)}\stepsTo\code{true}}
    \end{align*}
}
\spec{lengthLess}{int -> 'S language}{$\code{n}\geq\code0$}{\code{lengthLess n} is a total function such that
    \begin{align*} 
        & \mc L(\code{lengthLess n}) \\
        &= \compSet{\code{s}\in\Vals(\code{Sigma list})}{\code{((List.length s)<n)}\stepsTo\code{true}} 
    \end{align*}
}
\spec{lengthGreater}{int -> 'S language}{$\code{n}\geq\code0$}{\code{lengthGreater n} is a total function such that
    \begin{align*}
        &\mc L(\code{lengthGreater n}) \\
        &= \compSet{\code{s}\in\Vals(\code{Sigma list})}{\code{((List.length s)>n)}\stepsTo\code{true}}
    \end{align*}
}
\clearpage
\section{Implementation}
\auxLib{Language.sml}{31}{50}
\clearpage


\section{Lemmas}
Throughout, \code{L,L1,L2,L3 : Sigma language} are total.

\begin{proposition}[Totality] \label{prop:totality}
    All values of the \code{Sigma language} type produced using the \code{Language} module methods are total:
    \begin{itemize}
        \item \code{everything} is total
        \item \code{nothing} is total
        \item For any value \code{v : Sigma list}, \code{(singleton v)} is total
        \item For any value \code{l : Sigma list list}, \code{(just l)} is total
        \item All the curried higher-order functions (\code{singleton}, \code{just}, \code{Or}, \code{And}, \code{Not}, \code{Xor}, \code{lengthEqual}, \code{lengthLess}, \code{lengthGreater}, and \code{str}) are all \textit{total} in the trivial sense: upon being supplied one argument, they evaluate to a value (a function expecting the next curried argument)
        \item If \code{L1,L2 : Sigma language} are total,
            \begin{itemize}
                \item \code{Or(L1,L2)} is total
                \item \code{And(L1,L2)} is total
                \item \code{Not(L1)} is total
                \item \code{Xor(L1,L2)} is total
            \end{itemize}
        \item For any $\code{n}\geq\code 0$, 
            \begin{itemize}
                \item \code{lengthEqual n} is total
                \item \code{lengthLess n} is total
                \item \code{lengthGreater n} is total
            \end{itemize}
        \item If \code{L : char language} is total, so too is \code{str L}
    \end{itemize}
\end{proposition}
\begin{proof}
    We prove several paradigmatic cases, and the others can be done similarly.

    Recall \code{just} is implemented as
    \auxLibIndent{Language.sml}{37}{38}{-20pt}
    So we can prove the totality of \code{(just l)} by structural induction on \code{l : Sigma list list}. The base case is immediate: for any value \code{s : Sigma list}, $\code{just [] s}\stepsTo\code{false}$, a value. Inductively assuming \code{just xs s} valuable, then we can see that \code{(just (x::xs) s)} is also valuable, since \code{(s=x)} is valuable and, if \code{(s=x)} evaluates to \code{false}, then 
        \[ \code{just (x::xs) s} \quad\stepsTo\quad \code{just xs s} \]
    which our IH tells us is valuable, completing the proof.


    Recall \code{And} is implemented as
    \auxLibIndent{Language.sml}{41}{41}{-20pt}
    so if \code{L1} and \code{L2} are total, then \code{(L1 s)} and \code{(L2 s)} are valuable, hence \code{(L1 s) andalso (L2 s)} is valuable, proving \code{And(L1,L2)} total.

    Taking for granted that \code{List.length} is total, it follows that the expression \code{(List.length s)=n} is valuable. Since this is the body of \code{lengthEqual n s}, we get that \code{lengthEqual n} is total.

    Proving the totality of \code{str L} from the totality of \code{L} is a straightforward consequence of the totality of \code{String.explode}.
\end{proof}
\begin{lemma}
    \[ \code{just} \qeq \code{(foldr Or nothing) o (map singleton)} \]
\end{lemma}
\begin{proof} 
    It suffices to show that for all values \code{l : Sigma list list},
    \[ \code{just l} \qeq \code{foldr Or nothing (map singleton l)} \]
    which we do by structural induction on \code{l}.

    \bcBox \code{l = []}. Pick arbitrary \code{s : Sigma list}. Then
        \begin{align*}
                &\code{just [] s}\\
                &\qeq \code{false} \tag{Defn. \code{just}}\\
                &\qeq \code{nothing s} \tag{Defn. \code{nothing}}\\
                &\qeq \code{(foldr Or nothing []) s} \tag{Defn. \code{foldr}}\\
                &\qeq \code{(foldr Or nothing (map singleton [])) s} \tag{Defn. \code{map}}
        \end{align*}
        Establishing that $\code{just []}\eeq\code{foldr Or nothing (map singleton [])}$.

    \ihBox Suppose for some \code{xs : Sigma list list} that
    \[ \code{just xs} \qeq \code{foldr Or nothing (map singleton xs)} \]
    Now pick some \code{x : Sigma list}. We'll show that
    \[ \code{just (x::xs)} \qeq \code{foldr Or nothing (map singleton (x::xs))} \]
    Pick arbitrary \code{s : Sigma list}.
    \begin{align*}
        &\code{just (x::xs) s}\\
        &\eeq \code{(s=x) orelse just xs s} \tag{Defn. \code{just}}\\
        &\eeq \code{(x=s) orelse just xs s} \tag{Symmetry of \code=}\\
        &\eeq \code{(Fn.equal x s) orelse just xs s} \tag{Defn. \code{Fn.equal}}\\
        &\eeq \code{(singleton x s) orelse just xs s} \tag{Defn. \code{singleton}}\\
        &\eeq \code{(singleton x s) orelse (foldr Or nothing (map singleton xs)) s} \tag*{\refIH}\\
        &\eeq \code{Or(singleton x, (foldr Or nothing (map singleton xs))) s} \tag{Defn. \code{Or}, \autoref{prop:totality}, higher-order totality of \code{map} and \code{foldr}}\\
        &\eeq \code{(foldr Or nothing ((singleton x)::(map singleton xs))) s} \tag{Defn. \code{foldr}, \autoref{prop:totality}, higher-order totality of \code{map}}\\
        &\eeq \code{(foldr Or nothing (map singleton (x::xs))) s} \tag{Defn. \code{map}}
    \end{align*}
    so we're done.
\end{proof}
\begin{corollary}
    \[ \code{nothing} \qeq \code{just []} \]
\end{corollary}

\begin{lemma} \label{lemma:boolAlg}
For any total \code{L,L1,L2,L3 : Sigma language}, the following equivalences hold
    \begin{itemize}
        \item $\code{And(L1,And(L2,L3))} \qeq \code{And(And(L1,L2),L3)}$
        \item $\code{And(L1,L2)} \qeq \code{And(L2,L1)}$
        \item $\code{Or(L1,Or(L2,L3))} \qeq \code{Or(Or(L1,L2),L3)}$
        \item $\code{Or(L1,L2)} \qeq \code{Or(L2,L1)}$
        \item $\code{And(L,everything)} \qeq \code{L} \qeq \code{And(everything,L)}$
        \item $\code{Or(L,nothing)} \qeq \code{L} \qeq \code{Or(nothing,L)}$
        \item $\code{Not(Not(L))} \qeq \code{L}$
        \item $\code{Or(L1,And(L1,L2))} \qeq \code{L1} \qeq \code{And(L1,Or(L1,L2))}$
        \item $\code{Or(L1,And(L2,L3))} \qeq \code{And(Or(L1,L2),Or(L1,L3)))}$\\
              $\code{And(L1,Or(L2,L3))} \qeq \code{Or(And(L1,L2),And(L1,L3)))}$
        \item $\code{And(L,Not L)} \qeq \code{nothing}$\\
              $\code{Or(L,Not L)} \qeq \code{everything}$
    \end{itemize}
\end{lemma}

\begin{lemma} 
    For all values $\code{n}\geq\code 0$,
    \[ \code{Not(lengthGreater n)} \qeq \code{Or(lengthEqual n,lengthLess n)} \]
\end{lemma}
\begin{proof}
    Pick arbitrary \code{s : Sigma list}, and let \code{m} denote the value of \code{List.length s}.
    \begin{align*}
        &\code{(Not(lengthGreater n)) s}\\
        &\eeq \code{not(lengthGreater n s)} \tag{Defn. \code{Not}, \autoref{prop:totality}}\\
        &\eeq \code{not((List.length s)>n} \tag{Defn. \code{lengthGreater}}\\
        &\eeq \code{not (m>n)} \tag{Defn. \code{m}}\\
        &\eeq \code{m=n orelse m<n} \tag{math}\\
        &\eeq \code{((List.length s)=n) orelse ((List.length s)<n)} \tag{Defn. \code{m}}\\
        &\eeq \code{(lengthEqual n s) orelse (lengthLess n s)} \tag{Defn. \code{lengthEqual} and \code{lengthLess}}\\
        &\eeq \code{(Or(lengthEqual n, lengthLess n)) s} \tag{Defn. \code{Or}, \autoref{prop:totality}}
    \end{align*}
    as desired.
\end{proof}
This proof can be modified to prove similar equivalences, e.g.
    \[ \code{Not(lengthLess n)} \qeq \code{Or(lengthEqual n,lengthGreater n)}. \]

\end{document}

